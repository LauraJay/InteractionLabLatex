\section{Introduction}\label{sec:Introduction}

The main idea of Virtual Reality (VR) is to ensure a totally immersive user experience and the first important milestone was the development of the first Head-Mounted-Display (HMD) in 1966 by Sutherland and Goertz. This HMD offered the possibility to walk around in each virtual scene and to look at the scene from every view point. Despite this possibility, a totally immersive experience requires user interactions with virtual objects like selecting, grabbing, manipulating, movement and indirect controlling via widgets, gestures and voice input in virtual scenes. Especially a realistic grabbing and positioning of these objects is required which should come close to human motion sequences and fits to the human cognition and experiences. These kind of immersive interaction methods should include further requirements as well. A high precision rate is unavoidable and must be guaranteed. Another condition is that the method offer an ergonomic usage which is not  tiring for the users and can be learned easily.  

The number of VR devices and applications increased heavily during this decade and the relevance of consumer and business VR applications rises steadily. Several suppliers currently offer various methods and devices for grabbing interactions. The most common are the \textit{Oculus}-HMD, \textit{HTC Vive}-HMD, \textit{HTC Vive}-Controller, data gloves and motion capturing systems for hand-tracking like the \textit{LeapMotion}-Controller. Most devices offer a system with two hand controllers at this state. These hand controllers enable the use of virtual menus and offer a robust hand tracking. The latter is urgently needed to implement common grabbing interactions. Due to this rising amount of VR devices, it is vital for the development and improvement of grabbing methods to evaluate their usability and performance in an adequate environment as well as to compare them with the state of art methods. 

In the following sections, we describe and discuss an interactive VR system to compare different controller-based grabbing methods to evaluate their usability, error rate and time consume. This VR system runs with the \textit{HTC Vive}-HMD and two \textit{HTC Vive}-Controllers. Thus, all offered grabbing methods are controller based which could be learned in a special learning room and evaluated in a predefined VR supermarket with all required measurement tools.

\subsection{Motivation}\label{sec:Motivation}
Currently there no interaction laboratory exists that compares different grabbing interaction methods. Similar laboratories \cite{lin2016towards}\cite{website:TU}\cite{website:steam} have been developed but they do not allow the user to compare different kind of grabbing methods. In case of \cite{website:steam} and \cite{website:TU}, it provides only some VR applications or devices to experience the methods in different scenarios. Another \cite{lin2016towards} laboratory evaluates natural grabbing methods without a credible evaluation.
  
This circumstance calls for a virtual laboratory where different interaction methods can be compared, demonstrated, or tested in adequate virtual test scenes. Furthermore, all tests and compares should be based on scientific standards to allow factual and useful results. Thus, this laboratory should provide all required measurement tools to start credible and scientific studies of pre-implemented grabbing methods. This helps to standardise the evaluations and to increase their comparability of common studies. Another achievement is that the user friendliness of interaction methods which are not tiring and do not destroy the immersion could be improved by researcher. 

Another aspect of those user-friendly methods is the increasing usability of VR applications which will yield to a higher consumer preference of devices with their implementation. Therefore, the profit of VR device suppliers will be squeezed and the relevance of those products will expand worldwide. 

In an educational context it could be used for demonstrations during lectures. This will give the lecturer an effective tool at hand to explain the importance of usability as well as the advantages and disadvantages of each grabbing method. Another aspect of this laboratory is to give students a tool for the technical realisation of interaction studies in virtual (or augmented) reality environments.

\subsection{Project Goal}\label{sec:Project Goal}
Hence, a virtual interface laboratory with an environment to test and compare grabbing methods will be be developed. It will offer a possibility to develop new ones and it provide a use for teaching purposes as well. Thus, one task is to develop sophisticated test scenes for testing the interaction methods. These scenes will implement test exercises in different difficulty levels and represent typical and well-known environments like shops. All relevant parameters for the evaluation of the methods will be measured automatically and saved in an output comma separated value file. The latter must be easily imported in common statistic tools. Additional the laboratory will provide digital questionnaires to evaluate the usability of each provided method as well as one that enquire the system relevant properties like motion sickness, immersion, and latency.

\newpage


















