\section{Self-Assessment} \label{sec:SA} %Laura: Das können wir auch anders nennen!!!
\todo[inline, color=red]{???}

\textcolor{red}{Laura:Hier müssen wir uns am besten eine gemeinsame Struktur überlegen, oder? Vielleicht beschreibt jeder was er gemacht und dann eine kurze Selbstrefelktion?}


\subsection{Anna Bolder} \label{sec:SAAnna}
\todo[inline, color=red]{Anna}



\subsection{Vera Brockmeyer} \label{sec:SAVera}
\todo[inline, color=red]{Vera}

\subsection{Britta Boerner} \label{sec:SABritta}
\todo[inline, color=red]{Britta}

\newpage
\subsection{Laura Anger} \label{sec:SALaura}
\todo[inline, color=yellow]{Laura}

\textbf{Working Ours:} \textcolor{red}{97} \\ \\
\textbf{Written Sections:} \ref{sec:StateOfTheArt},  \ref{sec:SOTAInteractions}, \ref{sec:System} and \ref{sec:Interactions} (subsections included) \\ \\
\textbf{Responsibilities:}\\ \\
Besides mandatory tasks, like for example doing research on specific problems and helping with the project management, I was the head of far range interactions. It should  be obvious that I put my main effort into the various interaction methods of the \textit{Interaction Lab}. As it turned out it was not wise to separate between the interaction methods into two sections and letting two people (B. Boerner and me) work on each part. This is caused by the overlap of the different interactions. For example the \textit{Wand Grab} (compare section~\ref{sec:WandGrab}), which is a CR interaction is very similar to the FR interactions \textit{Raycast} (compare section~\ref{sec:Raycast}) and \textit{Extendable Raycast} (compare section~\ref{sec:ExtendableRay}). \\
To cut a long story short, I implemented the FR as well as the CR interactions together with B. Boerner. A brief description of all completed, as well as one unfinished, methods can be found in section \ref{sec:Interactions}.\\
Together with Anna Bolder it I integrated the implemented interactions into the actual scenes (compare section~\ref{sec:VRLabor}) of the \textit{Interaction Lab}. We also took care of the visualisation of the tasks for the user. Therefore we created a display on the wall of the supermarket as well as next to the controller. On both of the displays the corresponding tasks are presented for the user. \\
I was also involved in creating the project video and all images for this documentation.\\

\textbf{Self-Reflexion:}\\ \\
All in all I felt like the project run smoothly. Of course I was confronted with unknown terrain, as I never had implemented interactions of any kind. Maybe this was why I first started to create one script per interaction method. In the end of the project I noticed, that both Raycast methods and the \textit{Wand Grab} work after the same rules, so I combined them in one script. If I had done that earlier I could have avoided a lot of trouble for me and Anna Bolder as we needed to integrate them in to all scenes, separately first.

\newpage

























