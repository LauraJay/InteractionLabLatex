\section{Self-Assessment} \label{sec:SA}
\todo[inline, color=red]{???}


\subsection{Anna Bolder} \label{sec:SAAnna}
\todo[inline, color=yellow]{Anna}

\textbf{Working Ours:} \textcolor{red}{116} \\ \\
\textbf{Written Sections:} \ref{sec:SOTALabor},  \ref{sec:VRLabor} (subsections included), \ref{sec:Menu}, \ref{sec:selfteaching} and \ref{sec:tasks} \\ \\
\textbf{Responsibilities:}\\ \\
My main tasks in this project was the menu as well as the displaying of different informations, selfteaching and tasks, for the user. \\
At the beginning I created a shared Unity project and added the Steam VR plug-in. I created different rooms for the testing of the close and far range interactions. Also I designed the learning room, described in \ref{sec:Learningroom}. With V. Brockmeyer I discussed the structure of the supermarket. Additional I researched the options to switch between the rooms and created the basis for the script \textit{TargetTest.cs}.\\
After the creation of the basic scenes I researched for a solution of an radial menu on the controller. I implemented this with the ``VRTK'' plug-in. To collect all functions behind the buttons in one script I created the script \textit{menu.cs}.
As soon as all interactions were implemented L. Anger and I integrated the interactions in the menu and all scenes. Here I helped to simplify the switching between different interactions with a ray. \\
In the last weeks I implemented the selfteaching together with V. Brockmeyer. This was time-consuming because the counter has to be increased or set on different parts of the project and the height of the canvas has to be set for every step.\\
With the knowledge of the selfteaching I and L. Anger implemented also the tasks in the supermarket. Here we had to add a board on the wall and to short the actual tasks texts. \\
Also I did the test surveys together with L. Anger.\\

\textbf{Self-Reflexion:}\\ \\
In my opinion the project ran quiet well. Everybody had their part of the project and we also helped each other.\\
I underestimated some parts of my responsibilities. First the selfteaching took much more time than I expected. There were a lot more steps to do than I thought. Second the combination of all interactions in one menu and scene was more complicated. We did not think about that they could affect each other or that parts of the interaction have to be loaded at the start of the scene to work correctly.\\

\newpage
\subsection{Vera Brockmeyer} \label{sec:SAVera}
\todo[inline, color=yellow]{Vera}


\textbf{Working Ours:} 170 \\ \\
\textbf{Written Sections: }\ref{sec:Introduction}(subsections included),  \ref{sec:measurement}, \ref{sec:pm} (subsections included), \ref{sec:Reflexion}, \ref{sec:Conclusion} \\ \\
\textbf{Responsibilities:}\\ 
My main responsibility was the planning, organisation and management of the project. This includes the making of all project plans and the definition of it as well as to monitor the project progress and to guarantee the compliance of the schedule. Another task was the management of the internal and external communication as well as representation. This includes the making of appointments for meeting and required facilities as well as the steadily summary of the current project state. However, the distribution of the work tasks was my responsibility, too.

Beside the project management, I supported Anna Bolder with the implementation of the VR environment. In detail, I build the supermarket scene with 3D assets from the Steam Store and did the research for the tasks. Furthermore, I wrote all texts of the tasks and self-teaching and supported Anna Bolder with the implementation of the ladder. However, my other main task was the implementation, integration and output of the measurement. This includes the making of an \textit{MS Office Excel} template to evaluate the measuring's.


\textbf{Self-Reflexion:}\\ \\
I assume that the project worked very well without big complications that afford a time consuming solution or plenty of unnecessary work. According to the planning, it turned out that I planned to much in detail at the beginning which costs a lot of time resources and results in a complete refactoring of all plans. Another planning failure was the lack of enough research time during the complete project. 

The communication works quite well beside some small periods in the beginning where I planned too much and long team meetings and one where should have been more team meetings in May to avoid little misunderstandings. Beside this, I assume that I distribute the work tasks efficiently with regards to the skills of the team members. The external communication worked quite well beside some intermediately state of knowledges due to meetings without me. Especially, the translation of the usability questionnaires in German and the parallel decision to run the whole application in English language because of the brief schedule. 

According to my task during the development of the VR environment, I implemented a working measuring with all required output files. Nonetheless, the architecture of it do not allow an easy extension of further interaction methods or scenes. 

Briefly, the project team worked very good together and solved each problem directly with focus on the entire application without rest on details. The results correspond to our expectations besides some little failures.
\subsection{Britta Boerner} \label{sec:SABritta}
\todo[inline, color=red]{Britta}

\newpage
\subsection{Laura Anger} \label{sec:SALaura}
\todo[inline, color=yellow]{Laura}

\textbf{Working Ours:} \textcolor{red}{125} \\ \\
\textbf{Written Sections:} \ref{sec:StateOfTheArt},  \ref{sec:SOTAInteractions}, \ref{sec:System} and \ref{sec:Interactions} (subsections included) \\ \\
\textbf{Responsibilities:}\\ \\
Besides mandatory tasks, like for example doing research on specific problems and helping with the project management, I was mainly working on the implementation of the interactions. At the beginning of the project we divided the interactions into CR and FR (compare section~\ref{sec:Interactions}). At that time I became responsible for the FR methods of the \textit{Interaction Lab}. As it turned out it was not wise to distinguish between those two categories strictly, the borders of the jurisdiction of B. Boerner (Head of CR interactions) and me became more and more blurred. That is why we ended up working on most of the interaction methods together. This was more expedient, because for example the \textit{Wand Grab} (compare section~\ref{sec:WandGrab}), which is a CR interaction is very similar to the FR interactions \textit{Raycast} (compare section~\ref{sec:Raycast}) and \textit{Extendable Raycast} (compare section~\ref{sec:ExtendableRay}). Because B: Boerner went into holidays rather at the end of the implementation phase, I implemented the \textit{Wand Grab} on my own. \\
Together with A. Bolder I integrated the implemented interaction methods into the actual scenes (compare section~\ref{sec:VRLabor}) of the \textit{Interaction Lab}. We also took care of the visualisation of the tasks for the user. Therefore we created a display on the wall of the supermarket as well as next to the controller. On both of the displays the corresponding tasks are presented for the user. \\
I was also involved in creating the project video and all images for this documentation.\\
A. Bolder and I ran a test survey with 5 participants, which is described in section \textcolor{red}{???}.

\textbf{Self-Reflexion:}\\ \\
All in all I felt like the project run smoothly and well planned. Of course I was confronted with unknown terrain, as I never had implemented interactions of any kind. Maybe this was why I first started to create one script per interaction method. In the end of the project I noticed, that both Raycast methods and the \textit{Wand Grab} work after the same rules, so I combined them in one script. If I had done that earlier I could have avoided a lot of trouble for me and Anna Bolder as we needed to integrate them in to all scenes, separately first. \\
But all in all the workflow was well structured, as anyone had her own work area. Whenever this areas collides (for example the integration of the interactions into our scenes), we managed that both responsible people worked together on this task.

\newpage

























