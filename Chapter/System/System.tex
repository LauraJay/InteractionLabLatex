\section{System}\label{sec:System}
\todo[inline, color=red]{???}

\subsection{VR Labor}\label{sec:VRLabor}
\todo[inline, color=red]{Anna}

\subsection{Controller Menu} \label{sec:Menu}
\todo[inline, color=red]{??}

\subsection{Interaction Methods}\label{sec:Interactions}
\todo[inline, color=yellow]{Laura}
Of course there were various different interaction methods required to make the \textit{Interaction Lab} suitable for the testing described and evaluated in section~\ref{sec:evaluation}. Also all interaction methods are implemented to realise the grabbing of virtual objects, there can be spereated in two categoeris: close- and far range interactions (compare section~\ref{sec:CloseRange} and~\ref{sec:FarRange}). The so called close range combines all interactions which can be used to pick up objects in the direct reach of the user. As it is not quit easy and natural to walk through a huge area in VR, it is common to have grabbing interactions, which allow the user to grab objects which are normally seen as out of his reach~\cite{VRBook}. Interaction methods allowing such an acting are called far range interactions. \\
In the following sections all available interaction methods of the \textit{Interaction Lab} are presented. To guarantee a better overview the are sorted into the two categories close range (compare section~\ref{sec:CloseRange}) and far range (compare section~\ref{sec:FarRange}). The interaction methods described in sections~\ref{sec:ControllerGrab} and~\ref{sec:DistantControllerGrab} use snapping. This technique is described in section~\ref{sec:snapping}. In addition to that the integration of the various interaction methods into the testing scene, as well as the actual supermarket scenes (compare section~\ref{sec:VRLabor}) are depicted in section~\ref{sec:integration}. \\
For a better understanding it should be mentioned, that all interaction methods can be controlled with the \textit{HTC Vive}-controller. The technique is rather the same for every grabbing interaction: it is started by pulling the trigger and ended by releasing it.

\subsubsection{Close Range Interactions}\label{sec:CloseRange}
\todo[inline, color=red]{Laura}
As indicated above the close range can be interpreted as a synonym for the natural interaction radius of the person. Due to this definition it is excluded that those interactions can be used outside an area, which the person can reach with his arm, or to be more precise: with the controller in his hand. 

%They differ mainly in the accuracy 
%wie eingangs kurz erwähnt--> erklren was close range; natürliche Schranke die reale Raum
%geplante und umgesetzte
\paragraph{Grab via Controller} \label{sec:ControllerGrab}
%Umgsetzt in script bla.

\paragraph{Grab via Controller with slitghly Distance} \label{sec:DistantControllerGrab}
\paragraph{Grab via Stick} \label{sec:StickGrab}
%In conrtrast do the Interaction method described in \ref{sec:DistantControllerGrab} this method can be used to grab very tiny objects. Thereby it is not needed that the target object is very isolated. To give the user such an high grade of accuracy a stick is added to the controller like shown on picture BLA.

\subsubsection{Far Range Interactions}\label{sec:FarRange}
\todo[inline, color=red]{Laura}
%geplante, umgesetzte und fehlerhafte Interaktionen

\paragraph{Raycast} \label{sec:Raycast}
\paragraph{Indirect Raycast} \label{sec:IndirectRaycast}
\paragraph{Raycast Head Mounted Display} \label{sec:RaycastHMD}
%Der vollständigkeithalber..
%konnte nicht integirert werden weil...

\subsubsection{Snapping} \label{sec:snapping}
\subsubsection{Integration to the system} \label{sec:integration}
%was bisher eher unterschwellig behandelt wurde ist, dass  the dfferent interaction methods are imple,emted in various scripts.der einfachheithalber und weil es sch heerausgestellt hat DAS SICH DIE rays ansonsten ausschließen sind rays zusammen.



\subsection{Self-Teaching} \label{sec:selfteaching}
\todo[inline, color=red]{??}

\newpage