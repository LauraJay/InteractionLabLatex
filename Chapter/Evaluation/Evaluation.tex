\section{Evaluation} \label{sec:evaluation}
\todo[inline, color=red]{Britta}

\subsection{Tasks} \label{sec:tasks}
In the supermarket (compare section~\ref{sec:supermarket}) is confronted with four tasks. In every task an object has to be selected and placed on a target area. If the correct object is placed, the target area will change the color to signal that the tasks is successful done. In the script \textit{TargetTest.cs}, which is added as a component to the target object, is recognized when the target object hits the target area. At this moment the texture of target area is changed and the measurement (compare section~\ref{sec:measurement}) is stopped. \\
In the first three tasks the user has to select different objects. Therefore the user can decide which method he/she uses to grab the object. The selected method should differ depending on the tasks. In these tasks objects far away as well as in the CR should be picked. \\
The tasks will be shown next to the controller similar to the selfteaching (compare section~\ref{sec:selfteaching}), which the user is already aware of. The placement of the tasks is shown in the following figure. 

\begin{figure}[H] 
	\center 
	\includegraphics[width=12cm]{Images/TaskContreoller.PNG}
	\caption[Task shown next to the Controller.]{Task shown net to the Controller.}
	\label{fig:taskC}
\end{figure} 

The text for the tasks are saved in a CSV-file, similar to the selfteaching, see section \ref{sec:selfteaching}. The implementation is also comparable to the selfteaching and implemented in the script \textit{showTasks.cs}.

\begin{figure}[H] 
	\center 
	\includegraphics[width=12cm]{Images/TaskWall_1.PNG}
	\caption[Additional Information on the Wall.]{Additional Information on the Wall.}
	\label{fig:taskW1}
\end{figure}

To give the user some more instructions a information board within the supermarket is established, see figure \ref{fig:taskW1}. For the first three tasks there are only shown some basic information.\\
The last tasks will be repeated with every available method. That means, that the user has to pick up the same object five times. This object is placed, so that the user could use CR as well as FR methods (compare section~\ref{sec:Interactions}. The methods are implemented in a fif order, which can be seen in table \ref{tab: OrderMethods}. The methods will already be activated as soon as the user presses start. \\

\begin{table}[h]
\centering
 \begin{tabular}{|c|c|}
  Number of subtask & Method  \\ \hline
  1 & Close Range: Touch Grab  \\
  2 & Close Range: Wand Grab  \\
  3 & Far Range: Extendable Ray  \\
  4 & Close Range: Proximity Grab  \\
  5 & Far Range: Raycast \\
   \end{tabular}
  \caption[Order of methods in the Last Tasks.]{Order of methods in the Last Tasks.}
	\label{tab: OrderMethods}
 \end{table}

To help the user to figure out what method is activated the name of the method will be shown on the information board as soon as  he/she is in the new scene (compare figure~\ref{fig:taskW2}). 

\begin{figure}[H] 
	\center 
	\includegraphics[width=12cm]{Images/TaskWall_2.PNG}
	\caption[\textit{Touch Grab} is activated.]{\textit{Touch Grab} is activated.}
	\label{fig:taskW2}
\end{figure}

For the last tasks a usability questionnaire needs to be answered by the user, after he/she finishes the task. 


\subsection{Measurement} \label{sec:measurement}

\subsubsection{Time and Precision Measurement}
\todo[inline, color=yellow]{Vera}
As mentioned in section \ref{sec:tasks}, several tasks must be done by the users which are designed to evaluate various aspects. Thus, three different sets of measurements could be saved automatically in a task. All results are stored in a comma separated value text file that is named after their subtask and stored directly into the unity project folder. These files could be imported in all common statistic applications like \textit{MS Office Excel}, for example. Hence, the provided template is a \textit{Excel} file with the required basic statistic computations and visualisation of the results. Here, the output files could be easily integrated in the designated table fields. Following this, the template computes all useful mean values and standard deviations of the measurements. In other cases, it calculates the percentagewise proportion like the commonness of a method use. Each calculation is displayed in a corresponding diagram. Further, the ones that visualise the mean values includes an error bar that depends on the standard deviation. If necessary, other static computation like a significance test needs to be extended because the calculation highly depends on the study conditions and are not predictable.

The first measuring set is made for the learning procedure and should evaluate the affordable learning time of a grabbing method. Therefore, their usage time is recorded during the complete learning process and saved in milliseconds when a user presses the stop button. 

Afterwards, the first task with its three subtasks is run. Here, the user's method preference of a close or far range task should be evaluated. Hence, the application measures the usage time of every selected method and saves the id of the current method when the target object is placed successfully on the target area. The listing of this measurement shows significantly which methods were preferred and if a user had comprehend their scope.  

Finally, the precision as well as the grabbing and positioning time is measured in the second task for every provided method. Therefore, the grabbing time is started when a user presses the start button and stopped when the target object is grabbed. At this point, the positioning time is started immediately and stopped when the object is placed on the target area. These time measuring should show the workload and where the users have problems with the method. Additional, the selecting and positioning error rate is recorded during the whole process. Thus, this recorded data gives conclusions about the precision. The success of this task is recorded and the use of the snapping mode, too. Hence, all relevant informations are provided to get all important conclusions about the applicability of each method. 

\subsection{Questionnaires} \label{sec:questionnaires}

To not only use measured values to evaluate the interaction methods but also use the participants input, the participants have to fill out two kinds of questionnaires. The first one is a usability questionnaire, the \textit{SUS: a 'quick and dirty' usability scale} \citep{sus}. This questionnaire consists of ten questions. It is given to the user after each task, so each user fills out the questionnaires five times, once for each interaction method. This is done so the usability of each method can be determined. The investigator has to fill in two extra questions which are a testID (one number assigned to every participant) and which method was used. This is done to be able to draw an conclusion based on the used method. \\

The second questionnaire is a simulator sickness questionnaire by Kennedy et al. \citep{ssq} which evaluates the general discomfort caused by the virtual reality application. The investigator has to fill in the testID corresponding to the testID used for the usability questionnaire. \\
The questionnaires can either be printed out and used in paper form or Google Forms can be used. When done in Google Forms, the result can be imported into \textit{MS Office Excel}. Templates are  provided as \textit{Excel} files with basic statistic calculations and visualisation. 
The evaluation formula for the simulator sickness questionnaire is integrated and can easily be applied to larger study groups. The result consists of four values: nausea, optic, disorientation and the total score. The total score has a range from 0 to 160.7, but a theoretical result of over 100 would be so dangerous that an ambulance should be called. \\
The evaluation formula for the usability questionnaire is also integrated. The result has a range from 0 to 100. 100 is considered to be a perfect system. Above 70 are systems with good usability and a score under 50 indicates huge deficit in usability. 




\newpage