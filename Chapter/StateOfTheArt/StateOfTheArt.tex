\section{State of the Art} \label{sec:StateOfTheArt}
\todo[inline, color=yellow]{Laura}
In the following sections the fundamentals and scientific knowledge of all parts of the \textit{Interaction Lab} will be described. Publications of similar projects are described in section~\ref{sec:SOTALabor}. Furthermore, an overview of methods of  (grabbing) interactions (compare section~\ref{sec:SOTAInteractions}), as well as their evaluation methods (compare section~\ref{sec:SOTAInteractionEvalutiom}) are given. 


\subsection{VR Labor}\label{sec:SOTALabor}
\todo[inline, color=red]{Anna}

\subsection{VR Grabbing-Interactions}\label{sec:SOTAInteractions}
\todo[inline, color=yellow]{Laura}
To create a immersive virtual environment it is, inter alia, necessary to make use of adequate grabbing methods \cite{Bowman}. As mixed reality was not taken into consideration for implementing the \textit{Interaction Lab}, all objects, the user can interact with, are completely virtual. \\
The interaction methods provided by the \textit{Interaction Lab} have no haptical feedback \cite{768179}, which would allow a conclusion about the surface quality of the gripped objects. Furthermore, they only use the hardware described in section \ref{sec:Vive} for the execution of the interactions. \\
Many of the common methods of interaction, which are available in the \textit{Interaction Lab} (compare section~\ref {sec:Interactions}) are explained in the paper by Bowman and Hodges \cite{Bowman}. In their explanations they also mentioned the  \textit{Go-Go} technique, which presents a conceivable extension of the laboratory.

\subsection{VR Grabbing-Interactions Evaluation}\label{sec:SOTAInteractionEvalutiom}
\todo[inline, color=red]{Britta}

\newpage
