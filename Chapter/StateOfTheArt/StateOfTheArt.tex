\section{State of the Art} \label{sec:StateOfTheArt}
\todo[inline, color=yellow]{Laura}
In the following sections the fundamentals and scientific knowledge of all parts of the \textit{Interaction Lab} will be described. Publications of similar projects are described in section~\ref{sec:SOTALabor}. Furthermore, an overview of methods of  (grabbing) interactions (compare section~\ref{sec:SOTAInteractions}), as well as their evaluation methods (compare section~\ref{sec:SOTAInteractionEvalutiom}) are given. 


\subsection{VR Labor}\label{sec:SOTALabor}
Due to the progress in the development of grabbing interactions, various evaluations of those interactions were needed and implemented. There are a few papers and other articles. In the following sections three different works will be described shortly. \\
In the year 1999 Pouryrev and Ichikawa categorized and evaluated different grabbing interactions in their paper \cite{POUPYREV199919}. In their project they used hand tracking to measure the position of the real hand and to visualise a virtual representation of the hand. The participants than had to select different simple test objects, like cubes, spheres and cylinders. Also a positioning task was available.\\
A few years later Lee et al. evaluated different raycast grabbing methods \cite{lee2003evaluation}. Lee at al. used a 3D mouse for tracking the position and orientation of the hand. With this mouse the participants selected spheres on various and random positions. Therefore the participants were ask not to change the position of their head.\\
In the last year Eriksson published his master thesis, which is relative similar to this project \cite{eriksson2016reaching}. Erikson used the \textit{Oculus Rift} \cite{website:Oculus} with the Oculus Touch to provide the tracking and visualisation. In this project the participants also had to complete tasks in a virtual shop. The tasks had similar requirements, related to the selection, but also refer to manipulation and translation of the objects. 

\subsection{VR Grabbing-Interactions}\label{sec:SOTAInteractions}
\todo[inline, color=yellow]{Laura}
To create an immersive virtual environment it is, inter alia, necessary to make use of adequate grabbing methods \cite{Bowman}. As mixed reality was not taken into consideration for implementing the \textit{Interaction Lab}, all objects, the user can interact with, are completely virtual. \\
The interaction methods provided by the \textit{Interaction Lab} have no haptical feedback \cite{768179}, which would allow a conclusion about the surface quality of the gripped object. Furthermore, they only use the hardware described in section \ref{sec:Vive} for the execution of the interactions. \\
Many of the common methods of interaction, which are available in the \textit{Interaction Lab} (compare section~\ref {sec:Interactions}) are explained in the paper by Bowman and Hodges \cite{Bowman}. In their explanations they also mentioned the  \textit{Go-Go} technique, which presents a conceivable extension of the laboratory.

\subsection{VR Grabbing-Interactions Evaluation}\label{sec:SOTAInteractionEvalution}
\todo[inline, color=yellow]{Britta}

To determine usability of computer based applications has been a problem from the start of the computer development. Many usability questionnaires are available but short and easy to use ones are preferred. A widely used usability questionnaire is the \textit{SUS: A quick and dirty usability scale}~\cite{6sus}. It has been developed in 1986 by John Brooke. The ten questions are phrased in such a way that the questionnaire can be used generally used for any digital application or even websites. 
Corresponding to that, since virtual reality applications started to appear, there needed a way to test how it effected the users. Kennedy et al. developed a \textit{Simulator sickness questionnaire} in 1993~\cite{ssq}, which is still used today. It tests 16 symptoms the user might feel and give a total score as the result as well as three sub results for nausea, oculomotor and disorientation. Even though these questionnaires are not the newest, the frequent use of them show that they are still up to date. 

\newpage
