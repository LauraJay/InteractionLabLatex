\section{Reflecion} \label{sec:Reflexion}
\todo[inline, color=yellow]{Vera}

Finally, a running and nearly complete application was developed during this project.
The test environments in form of the learning room and supermarket are working without a latency or blackouts. They provide an effective virtual laboratory that could be used by researchers or for educational proposes. Both environments integrate five perfectly working grabbing interaction methods. First, the close range methods Touch, Proximity and Wand Grab are implemented and the first two extends an optional snapping mode. Second, the two far range methods Raycast and Extendable Ray are successfully integrated into the application.
 
All measurements of the tasks are saved automatically in output files that could be evaluated with the provided \textit{MS Office Excel} template. The usability questionnaire forms could be accessed by every mobile device or computer and their results are saved automatically as well. They are offered in English and German language. Certainly, they could be evaluated with the template, too. Additional, all required consent forms are delivered in both languages with the application as PDF file.

As mentioned in section \ref{sec:PMProblems}, the HMD Raycast and Gogo method could not be realised but unfortunately there are more goals that are not yet achieved or do not work completely without faults. First, currently the self teaching mode only works if the introductions were strictly followed. In case of a mistake, it will recover only when a new method is selected. However, there is no reminder of the usability questionnaires during the last task for the study supervisor. Therefore, he is forced to be totally focused during a test. Another failure is the missing question of the VR experience in these questionnaires and the lack of the general input of the test id in the application.

There are some technical mistakes as well which are not yet fixed. A main problem is the potential loss of script links in the unity project when the laboratory is changed or a git repository is pulled. A further problem is the lack of an easy method or scene extension in the application due to the complex measuring and interaction script architecture. Other small and rare troubles are the visible ray in front of transparent surfaces or object that fall sometimes through the ground.

\newpage

























