\section{Project Management} \label{sec:pm}
\todo[inline, color=red]{Vera}

This chapter describes the project planning and management of the \textit{Interaction Lab}. It is divided into the different project phases. Each division includes all important facts of its project period.

\subsection{Project Definition} \label{sec:PMProjectdefinition}
\todo[inline, color=red]{Vera}
This section describes the results of the project definition phase in detail. This includes a problem analysis, a list of objectives and requirements, a solution concept as well as a workability analysis.

\subsubsection{Problem Analysis}\label{sec:PMProblemAnalysis}
\todo[inline, color=red]{Vera}
The demand for Virtual Reality (VR) devices and applications increased heavily since the first consumer devices like \textit{HTC Vive} and \textit{Oculus Rift} were released during last years. One main difficulty of the current development of VR-applications is the lack of standardisation of the Software Development Kit (SDK) and interfaces. The most acknowledged suppliers \textit{HTC} and \textit{Oculus} do not work together or force standards for VR application development. Thus, all applications are system related and incompatible with other devices. Accordingly, each device offers different opportunities of interaction methods. These methods could be divided in the acknowledged categories selecting, grabbing, manipulating, movement and indirect controlling via widgets, gestures and voice input. Several suppliers currently offer different devices for interaction. And with focus on the grabbing and positioning methods, the most common are the \textit{Oculus}-HMD, \textit{HTC Vive}-HMD, \textit{HTC Vive}-Controller, data gloves and motion capturing systems for hand-tracking like the \textit{LeapMotion}-Controller.

As mentioned in section \ref{sec:Motivation}, there exist currently no interaction laboratory which compares the different interaction methods in a scientific and credible way. Hence, the development of a virtual laboratory is highly requested to compare and test different interaction methods in adequate test environments. Thus, user friendly interaction methods which nearly full-fill usability requirements could be improved by researcher which yields to a higher demand of VR devices and application. That will squeeze the profit of VR device suppliers which include those user friendly interaction methods. 

\subsubsection{Usage Context}\label{sec:PMUsageContext} 
\todo[inline, color=red]{Vera}
Hence, the required laboratory has mainly two usage contexts. First, it could be used to run scientific studies in VR research or development. Second, it could demonstrate and exemplify the differences of grabbing interactions in education proposes or support the students to develop and test grabbing methods on their own during lectures. 

\subsubsection{Objective and Requirements}\label{sec:PMRequirements}
\todo[inline, color=red]{Vera}

At least two scenes should be realised to provide a laboratory which allows to run scientific and reliable study as well as is useful for the education of students. In the first scene, the user will be able to learn the offered grabbing methods. Therefore, this room is will provide simple cubes of various sizes which are in different distances from the users. Every cube is moveable and could be placed at every place. Each user is forced to follow the introductions of a self teaching before every offered method could be tested independently. The current user can only begin with the actual study after every method is trained to ensure equal preconditions.

The second room will be modelled after a supermarket because this model offers various options of grabbing and positioning tasks. In this room, the participant will get different tasks which will differ by complexity, distance of grabbing and size of the objects. The user will be able to change the options of grabbing independently but not choose the current method. An optional extension of the project will be another type of task where the user decides which type of method is preferred for this task.

The grabbing methods can be categorised into close range and far range and include the grabbing, rotating, and positioning of an object. Possible types of close range methods, are the actual touching of a movable object to select it or by holding the controller in the proximity of it but without touching it. Another more precise option is the selection with a thin wand in front of the controller. This of collection of methods that includes close human cognition methods as well as less or very accurate ones. 
The far range interaction will have different options as well. One will be a ray that shoot out of the controller, another one will extend a ray from the head and the third one will extend the arm in the pointed direction. This means the user will be able to point at an object with the controller or to look in the direction of it.

The system offers two measurements and the related saving of the different parameters. First, the duration time is measured for every performed task to compare and validate the performance of the different interaction methods. Second, every single grabbing try of a task will be counted and saved to get a conclusion about the learn-ability, accuracy, and performance.

Furthermore, there will be a questionnaire designed to give the users of \textit{Interaction Lab} an usability evaluation tool at hand. This questionnaire will test parameters as tiring, learnability, self-descriptiveness and fulfilling expectations.


\subsubsection{Solution Concept}\label{sec:Projectconcept}
\todo[inline, color=red]{Vera}
An interaction laboratory for grabbing and positioning interactions at close or far range will be developed in \textit{Unity}. It includes two test rooms e.g. scenes, where the first is a learning room, in which the users can get familiar with the interaction methods. The second room is designed as a supermarket. This environment was chosen because it offers various possibilities of exercises under changing difficulties like grabbing small mushrooms, fetching distantly placed tins or putting goods on provided target areas. The exercises are offered in form of a tasks that tells the participant what goods have to be grabbed and repositioned. These various tasks are predefined and cover all difficulties that a type of grabbing method could have. They are displayed on tables which are connected to the controller and could be shown or hide in the controller menu. 

All rooms are implemented in Unity and the VR components are controlled by the same framework. Further, the \textit{HTC Vive}-HMD and the corresponding controllers are used to run the interactions, imaging and orientation in the environment. It is planned to realise at least six interaction methods of grabbing and positioning. Additional, the complete framework should be compatible with new test scenes and other interaction categories. 

The system offers a measurement of the accuracy as well. A time measuring of duration and an error rate for every performed task is planned. Each measuring of every room is automatically saved in an output file which could be easily imported in common statistic tools. Furthermore, there will be a questionnaire designed to give the users a usability and simulator sickness evaluation tool at hand. This usability questionnaire will test parameters as tiring, learn-ability, self-descriptiveness and fulfilling expectations of each method. Whereas the simulator sickness evaluation asks for motion sickness and other system properties of the complete system. All questionnaires are acknowledged and pre-tested questionnaire which fits the requirement of VR systems and application. The results of each questionnaire will be saved in an output file as well.
\subsubsection{Workability Analysis}\label{sec:PMworkabilityAnalysis}
There are several risks according to the concept in section \ref{sec:Projectconcept}. First, the measurements could be implemented incompletely or inaccurately. This can be avoided by a thorough testing before the final release with some external test persons. The tasks could be incomprehensible for them as well which should be prove as well.
The system integration future extensions could cause trouble. Therefore, the systems architecture should be designed wisely and consequently to avoid incompatibilities. Another risk of the implementation is that they might be more costly and complex as recommended but this is widely acknowledged. After the implementation is finished the interaction method performance or validation could be too expensive which results in a higher latency. These circumstances must be observed during the implementation and testing. Due to the high workload of the testing, the time slot for it and the trouble shooting might be underestimated. Another time risk is that there is limited access to the facilities and VR laboratory because of the huge number of running project at the current time.

Nevertheless, the concept is feasible and the project goals could be reached during the time schedule because all the risk seems to manageable and could be observed during the scheduled testing.

The demand of the students project are satisfied and a financial profitability check is not necessaries due to the fact that the facilities of the university can be used and no further purchases are affordable.


\subsubsection{Project Organisation}\label{sec:PMProjectOrganisation}
\todo[inline, color=red]{Vera}

The project manager is Vera Brockmeyer who mainly should manage the appointments and facilities as well as to communicate to the outside. The latter is done via email or in a meeting with the concerned persons. Another task is to create and maintain the project plans that includes to keep the overview of the complete project progress and to ensure the milestones.
The current state should be hand out weekly to the team in form of an email or a team meeting.

All other team members have their own responsibilities. Anna is head of the scene building which includes the definition of the general scene design, research, and to inform the project manager about current problems and timing. The latter two points concern each head of a section. The other section is split into the close and far range interactions. The head of close range interaction is Britta Boerner and the other is Laura Anger. Both manage the implementation of their section.

The formal an informal non-verbal communication in the team is done via email with the subject \textit{VR Interface Lab} and a \textit{Google Calendar} is exclusively maintained by the project manager where all team appointments are intercalated. This calendar shows the availability of all team members and the VR laboratory, too. More complex problems or team decisions are made in the weekly team meeting with stringently required appearance. Due to the requirements and availability of the team members, the meeting is held via \textit{Skype} or in personal.

All files belonging to the project are organized in a cloud folder of \textit{Google Drive} or in two \textit{GitHub} repositories. The first one is for all \textit{LaTex} files and the second manage the complete framework. Whereas the cloud folder contains presentation files, graphics and images, To-Do-List, papers and more.

The required facility is one of the VR laboratory of the faculty which should have a minimum size of $15$qm and be located in the university building. These laboratories have a complete \textit{HTC Vive}
system and a compatible computer (section \ref{}).




\subsection{Project Planning} \label{sec:ProjectPlanning}
\todo[inline, color=red]{Vera}

\subsubsection{??}\label{sec:??}
\todo[inline, color=red]{Vera}

\subsubsection{??}\label{sec:??}
\todo[inline, color=red]{Vera}




\subsection{Project Execution} \label{sec:ProjectExecution}
\todo[inline, color=red]{Vera}

\subsubsection{??}\label{sec:??}
\todo[inline, color=red]{Vera}

\subsubsection{??}\label{sec:??}
\todo[inline, color=red]{Vera}


\subsection{Project Completion}
\todo[inline, color=red]{Vera}

\subsubsection{??}\label{sec:??}
\todo[inline, color=red]{Vera}

\subsubsection{??}\label{sec:??}
\todo[inline, color=red]{Vera}


\newpage

























