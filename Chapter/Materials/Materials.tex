\section{Materials}\label{sec:Materials}
The following section will give an overview over all hard- and software that was used in this project. 

\subsection{Hardware}\label{sec:Hardware}
The hardware consists of a head-mounted display, the HTC Vive and a computer. 

\subsubsection{Computer}\label{sec:Computer}

\begin{table}[H]
	\centering
	\begin{tabular}{|l|l|}
		\hline
		\Absatzbox{}
		\textbf{CGPC6}& \textbf{Beschreibung} \\
		\hline
		Prozessor & Intel Core i7 6700 CPU @ $4\times3.4-4.0\,$GHz \\
		\hline
		Arbeitsspeicher & $16\,$GB \\
		\hline 
		Grafikkarte & NVIDIA GeForce GTX 980\\
		\hline
		Betriebssystem & Windows 10 Education 64 bit \\
		\hline
		Schnittstellen & $2\times$ USB 3.0, $5\times$ USB 2.0, $1\times $ HDMI\\
		\hline
	\end{tabular}
	\caption{Summary of the technical specifications of the used computer for the \textit{Unity} simulation.}
	\label{tab:Computer}
\end{table}

The hard- and software requirements for using \textit{Unity} with the \textit{HTC Vive}, which are listed in Table~\ref{tab:viveReq}, are being exceeded by the used computer.   

\begin{table}[H]
	\centering
	\begin{tabular}{|l|l|}
		\hline
		\Absatzbox{}
		\textbf{HTC Vive}& \textbf{System requirements} \\
		\hline
		Prozessor & mindestens Intel Core i5 4590 oder AMD FX 8350\\
		\hline
		Grafikkarte & mindestens NVIDIA GeForce GTX 1060\\
		&oder AMD Radeon RX 480\\
		\hline
		Arbeitsspeicher & mindestens $4\,$GB\\		
		\hline
		Videoausgang & $1\times$ HDMI 1.4 Anschluss oder DisplayPort 1.2\\
		\hline
		USB & $1\times$ USB 2.0-Anschluss\\
		\hline
		Betriebssystem & Windows 7 SP1, Windows 8.1 oder Windows 10\\
		\hline
	\end{tabular}
	\caption{\textit{HTC Vive} System requirements~\cite{website:HTC_Vive_Ready}.}
	\label{tab:viveReq}
\end{table}

\subsubsection{HTC Vive}\label{sec:Vive} 
The \textit{HTC Vive} is a head-mounted display which is being produced by \textit{HTC} in cooperation with \textit{Valve}~\cite{website:Valve}. It was introduced on the 1st of March 2015 at the \textit{Mobile World Congress}~\cite{website:mobileworldcongress}.The resolution of the integrated display is $2160\times1200$\,pixels which equals $1080\times1200$\, pixels per eye. The head-mounted display offers a field of vision of $110^\circ$ with a refresh rate of $90\,Hz$ \cite{website:HTC_Vive}.
The system requirements can be seen in table \ref{tab:viveReq}. \\ \textcolor{red}{Hier wird dreimal "the" als Satzanfang benutzt.}

To determine the head-mounted displays position within the room, the lighthouse technology~\cite{website:Lighthouses} by \textit{Valve} is used. \textcolor{red}{Evtl. Furthermore, the lighthouse technology by .... used to determine ... room.} Additional, the head-mounted display is equipped with a gyroscope, an acceleration sensor and a laser position measurer. The interaction with virtual objects is made possible with a proprietary hand controller. 
	
	\subsection{Software}
	
	\subsubsection{Unity}\label{sec:unity}
	\textit{Unity} is a so called game engine, meaning a developing and runtime environment, which was created for the development of 3D games. The software was released on the 6th of June 2005 \cite{haas2014histor} and is being developed and marketed by \textit{Unity Technologies} \cite{website:Unity}. Unity is widely used in the game development industry. $34\,\%$ of the Top~1000 mobile free to play games are developed with \textit{Unity} \cite{website:UnityPR}. \textcolor{red}{Kann es sein dass dieser Satz keinen Sinn ergibt?}
	
	\textit{Unity} offers broad platform support \cite{website:UnityMultiPlatform} and also allows the development for head-mounted display, like the \textit{Oculus Rift} \cite{website:UnityVRoverview} or the \textit{HTC Vive} \cite{website:UnityVRoverview}, which was used in this project.
	
	\subsubsection{Visual Studio 2015}\label{sec:VisualStudio}
	\textit{Micosoft Visual Studio 2015} is a widespread development environment (IDE). It supports the programming languages Visual Basic, Visual C$\#$, and Visual C++ amongst others. With the help of the IDE a developer can program and compile Win32/ Win64 and web applications. For \textit{Interaction Lab} version~$14.0.\-25123.00$ update $2$ was used. \textcolor{red}{Hier fehlt noch die Referenz. Ich glaube ich habe damals einfach auf die Homepage verwiesen.}
	
	\subsubsection{Steam VR}
	\textit{Steam VR} \cite{website:steamVR} is an interface between the \textit{HTC Vive} and \textit{Unity}. \textit{Steam VR} has to be installed on the computer to use the head-mounted display (HMD). A small GUI element can be seen on the monitor, which show the current status of the \textit{HTC Vive}. It communicates error messages and is used for the system calibrating. \\
	Within \textit{Unity}, \textit{Steam} provides a plug-in, which can be embedded into the \textit{Unity} scenes. The developer could integrate the VR options easily per drag-and-drop into a existing \textit{Unity} scene. 
	The provided \textit{Unity} prefab includes all necessary elements to communicate with the hardware. A determination of position and a camera rig for the stereoscopic display is provided as well as the possibility for hand controller input and further use of this data.
	
	\subsubsection{Virtual Reality Toolkit (VRTK)}\label{sec:VRTK}
	\todo[inline, color=yellow]{Anna}
	%\textcolor{red}{@Anna: Hier kurz ein bisschen erklären was die VRTK ist und wofür du sie benutzt hast.}
The ``VRTK'' (Virtual Reality Toolkit) is a free packet with different scripts and prefabs for interaction in the virtual reality \cite{asset_VRTK} \cite{VRTK}. In this project different parts of this plug-in are used.\\ 
The \textit{radial menu} script is used for the controller menu, see section \ref{sec:Menu}. This \textit{radial menu} creates menu buttons over the touchpad of the \textit{HTC Vive} controller. By scrolling over the touchpad a button can be selected. The selection will be shown by highlighting the button. If this button should be executed, the touchpad has to be pressed when the button is selected. The number of buttons per menu can be changed within the \textit{Unity} editor and also each button can get a different icon. Through a link to a function of a script this function can be executed if the button is pressed.
Furthermore the prefab for \textit{controller tips} is used for visualisation of the selfteaching(compare section \ref{sec:selfteaching}, and the tasks, compare section \ref{sec:tasks}. This prefab creates a area next to the controller and a line from this area to the selected button of the controller. This area is parented to the controller and moves according to it. Different settings can be changed within the \textit{Unity} editor or by a script.\\
Besides that there is a very similar prefab, the \textit{object tips}. This is also a canvas, in which information can be displayed, but this can be fixed on any object you like. This prefab is used for the labelling of the target object as well as the visualisation of additional informations in the supermarket, see section \ref{sec:tasks}.

	\newpage
