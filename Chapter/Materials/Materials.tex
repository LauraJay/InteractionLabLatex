\section{Materials}\label{sec:Materials}
\todo[inline, color=red]{Britta}

\subsection{Hardware}\label{sec:Hardware}
\todo[inline, color=red]{Britta}


\subsubsection{Computer}\label{sec:Computer}
\todo[inline, color=red]{Britta}
\textcolor{red}{@Britta: Vielleicht möchtest du die Tabelle einfach übernehmen und nur die Daten des PCs ändern?}\\

\begin{table}
	\centering
	\begin{tabular}{|l|l|}
		\hline
		\Absatzbox{}
		\textbf{CGPC6}& \textbf{Beschreibung} \\
		\hline
		Prozessor & Intel Core i7 6700 CPU @ $4\times3.4-4.0\,$GHz \\
		\hline
		Arbeitsspeicher & $16\,$GB \\
 		\hline 
		Grafikkarte & NVIDIA GeForce GTX 980\\
		\hline
		Betriebssystem & Windows 10 Education 64 bit \\
		\hline
		Schnittstellen & $2\times$ USB 3.0, $5\times$ USB 2.0, $1\times $ HDMI\\
		\hline
	\end{tabular}
	\caption[Übersicht technische Daten des Computers für \emph{Unity}-Simulation]{Übersicht der technischen Daten des Computers für die \emph{Unity}-Simulation.}
	\label{tab:Computer}
\end{table}

Die Hard- und Software-Voraussetzungen für die Ausführung der \textit{Unity}-Anwendung in Verbindung mit der \textit{HTC Vive}, welche in Tabelle~\ref{tab:viveReq} aufgelistet sind, werden von dem verwendeten Computer übertroffen.

\begin{table}
	\centering
	\begin{tabular}{|l|l|}
		\hline
		\Absatzbox{}
		\textbf{HTC Vive}& \textbf{Systemvoraussetzungen} \\
		\hline
		Prozessor & mindestens Intel Core i5-4590 oder AMD FX 8350\\
		\hline
		Grafikkarte & mindestens NVIDIA GeForce™ GTX 1060\\
		&oder AMD Radeon™ RX 480\\
		\hline
		Arbeitsspeicher & mindestens $4\,$GB\\		
		\hline
		Videoausgang & $1\times$ HDMI 1.4-Anschluss oder DisplayPort 1.2\\
		\hline
		USB & $1\times$ USB 2.0-Anschluss\\
		\hline
		Betriebssystem & Windows 7 SP1, Windows 8.1 oder Windows 10\\
		\hline
	\end{tabular}
	\caption[\emph{HTC Vive} Systemvoraussetzungen]{\emph{HTC Vive} Systemvoraussetzungen~\cite{website:HTC_Vive_Ready}.}
	\label{tab:viveReq}
\end{table}




\subsubsection{HTC Vive}\label{sec:Vive} 
\todo[inline, color=red]{Britta}
\textcolor{red}{@Britta: Vielleicht möchtest du das nur auf Englisch übersetzen?}\\
Die \textit{HTC Vive} ist ein Head-Mounted Display, welches von \textit{HTC} in Kooperation mit \textit{Valve}~\cite{website:Valve} produziert wird. Vorgestellt wurde dieses am 1.\ März 2015 im Vorfeld des \textit{Mobile World Congress}~\cite{website:mobileworldcongress}.\\
Die Auflösung des Displays beträgt insgesamt $2160\times1200$\,Pixel, was $1080\times1200$\,Pixeln pro Auge enstpricht. Die Brille bietet ein Sichtfeld von bis zu $110^\circ$ bei einer Bildwiederholrate von $90\,Hz$ \cite{website:HTC_Vive}. Alle technischen Systemvoraussetzungen können in Tabelle \ref{tab:viveReq} eingesehen werden. \\
Zur Positionsbestimmung im Raum wird die Lighthouse-Technologie~\cite{website:Lighthouses} von \textit{Valve} genutzt. Zusätzlich sind neben einem Gyroskop auch ein Beschleunigungssensor und ein Laser-Positionsmesser verbaut. Mittels proprietärer Hand-Controller wird bei der \emph{HTC Vive} eine Interaktion mit virtuellen Objekten ermöglicht.








\subsection{Software}
\todo[inline, color=red]{Britta}


\subsubsection{Unity}\label{sec:unity}
\todo[inline, color=red]{Britta}
\textcolor{red}{@Britta: Vielleicht möchtest du das nur auf Englisch übersetzen?}\\

\emph{Unity} ist eine sogenannte Spiel-Engine, also eine Entwicklungs- und Laufzeitumgebung, die speziell auf die Entwicklung von 3D-Spielen ausgelegt ist. Die Software wurde am 6. Juni 2005 veröffentlicht \cite{haas2014history} und wird von \textit{Unity Technologies} \cite{website:Unity} entwickelt und vertrieben. In der Spieleentwicklung ist \textit{Unity} weit verbreitet, so werden beispielsweise $34\,\%$ der kostenfreien Top-1000-Spiele im mobilen Sektor mit \textit{Unity} entwickelt \cite{website:UnityPR}.

\emph{Unity} bietet eine sehr breite Plattformunterstützung \cite{website:UnityMultiPlatform} und erlaubt ebenso die Entwicklung für Head-Mounted Displays, wie etwa die \textit{Oculus Rift} \cite{website:UnityVRoverview} oder auch die in diesem Projekt verwendete \textit{HTC Vive} \cite{website:UnityVRoverview}.


\subsubsection{Visual Studio 2015}\label{sec:VisualStudio}\todo[inline, color=red]{Britta}
\textcolor{red}{@Britta: Vielleicht möchtest du das nur auf Englisch übersetzen?}

\textit{Micosoft Visual Studio 2015} ist eine verbreitete integrierte Entwicklungsumgebung (IDE), welche unter anderem die Programmiersprachen Visual Basic, Visual C$\#$, und Visual C++ unterstützt. Mit Hilfe dieser IDE kann ein Entwickler Win32/ Win64 Anwendungen sowie Web-Applikationen und Webservices \cite{website:VisuStud} programmieren und anschließend kompilieren. Für \textit{MArC} wurde mit der Version~$14.0.\-25123.00$ Update $2$ gearbeitet.




\subsubsection{Steam VR}
\todo[inline, color=red]{Britta}
\textcolor{red}{@Britta: Vielleicht möchtest du das nur auf Englisch übersetzen?}\\

\textit{Steam VR} \cite{website:steamVR} ist die Schnittstelle zwischen der \textit{HTC Vive} und \textit{Unity}. Um das HMD nutzen zu können, musss \textit{Steam VR} auf dem Computer installiert sein. Für den Nutzer ist ein kleines GUI Element auf dem Monitor sichtbar, welches den Status der Geräte der \textit{Vive} darstellt. Hierdurch werden Fehlermeldungen kommuniziert, Kalibrierungen durchgeführt und eine Kommunikation mit dem HMD bereitgestellt, so dass das Gerät im Fall der Fälle neu gestartet werden kann.\\
Innerhalb von \textit{Unity} stellt \textit{Steam} ein Plugin zur Verfügung, welches direkt in Szenen in \textit{Unity} eingebettet werden kann. Der Entwickler ist also in der Lage, eine vorhandene \textit{Unity}-Szene um die VR Möglichkeit bequem per Drag-and-drop-Technik zu erweitern.\\
Das bereitgestellte \textit{Unity}-Prefab beinhaltet alle notwendigen Elemente um mit der Hardware kommunizieren zu können. Dabei wird eine Positionsbestimmung ebenso wie ein Kamera Rig für die stereoskopische Bildwiedergabe bereitgestellt, wie auch die Controllereingabe und Weiterverwendung der Daten möglich gemacht.


\newpage
